%%%%%%%%%%%%%%%%%%%%%%%%%%%%%%%%%%%%%%%%%%%%%%%%

% This Latex file will act as template to help you make assignments or submit homework.

% Every .tex file usually consists of four parts.
% 1. Document Class
% 2. Packages
% 3. Header
% 4. Your Document

%%%%%%%%%%%%%%%%%%%%%%%%%%%%%%%%%%%%%%%%%%%%%%%%
% 1. Document Class
%%%%%%%%%%%%%%%%%%%%%%%%%%%%%%%%%%%%%%%%%%%%%%%%
 
% The first command you will always have will declare your document class. This tells LaTeX what type of document you are creating (article, presentation, poster, etc). 
% \documentclass is the command
% in {} you specify the type of document
% in [] you define additional parameters
 
\documentclass[a4paper,11pt]{article} % This defines the style of your paper

%%%%%%%%%%%%%%%%%%%%%%%%%%%%%%%%%%%%%%%%%%%%%%%%
% 2. Packages
%%%%%%%%%%%%%%%%%%%%%%%%%%%%%%%%%%%%%%%%%%%%%%%%

% Packages are libraries of commands that LaTeX can call when compiling the document. With the specialized commands you can customize the formatting of your document.

% First, we usually want to set the margins of our document. For this we use the package geometry. We call the package with the \usepackage command. The package goes in the {}, the parameters again go into the [].
\usepackage[top = 2.5cm, bottom = 2.5cm, left = 2cm, right = 2cm]{geometry} 

% Unfortunately, LaTeX has a hard time interpreting German Umlaute. The following two lines and packages should help. If it doesn't work for you please let me know.
% \usepackage[T1]{fontenc}
% \usepackage[utf8]{inputenc}

% The following two packages - multirow and booktabs - are needed to create nice looking tables.
\usepackage{multirow} % Multirow is for tables with multiple rows within one cell.
\usepackage{booktabs} % For even nicer tables.

% As we usually want to include some plots (.pdf files) we need a package for that.
% \usepackage{graphicx} 

% The default setting of LaTeX is to indent new paragraphs. This is useful for articles. But not really nice for homework problem sets. The following command sets the indent to 0.
\usepackage{setspace}
\setlength{\parindent}{0in}
\usepackage{hyperref}

% Package to place figures where you want them.
\usepackage{float}

% The fancyhdr package let's us create nice headers.
\usepackage{fancyhdr}

% For simple, non "graphical" tree you can use the dirtree package:
\usepackage{dirtree}

% For highlighting phrases using graphical boxes.
\usepackage[utf8]{inputenc}

% for math functions
\usepackage{amsmath}

%for coloring the text
\usepackage[dvipsnames]{xcolor}
\usepackage{enumitem}
\usepackage{listings}
\usepackage{color}
\usepackage{tikz}
\usepackage[utf8]{inputenc} 
\usepackage[T1]{fontenc}
\lstdefinestyle{customasm}{
    belowcaptionskip=1\baselineskip,
    frame=single, 
    frameround=tttt,
    xleftmargin=\parindent,
    language=[x86masm]Assembler,
    basicstyle=\footnotesize\ttfamily,
    commentstyle=\itshape\color{green!60!black},
    keywordstyle=\color{blue!80!black},
    identifierstyle=\color{red!80!black},
    tabsize=4,
    numbers=left,
    numbersep=8pt,
    stepnumber=1,
    numberstyle=\tiny\color{gray}, 
    columns = fullflexible,
}

%%%%%%%%%%%%%%%%%%%%%%%%%%%%%%%%%%%%%%%%%%%%%%%%
% 3. Header (and Footer)
%%%%%%%%%%%%%%%%%%%%%%%%%%%%%%%%%%%%%%%%%%%%%%%%

% To make our document nice we want a header and number the pages in the footer.

\pagestyle{fancy} % With this command we can customize the header style.

\fancyhf{} % This makes sure we do not have other information in our header or footer.

\lhead{\footnotesize CSO: Lab Exam}% \lhead puts text in the top left corner. \footnotesize sets our font to a smaller size.

%\rhead works just like \lhead (you can also use \chead)
% \rhead{\footnotesize Lastname 1, Lastname 2 (\& Lastname 3)} %<---- Fill in your lastnames.

% Similar commands work for the footer (\lfoot, \cfoot and \rfoot).
% We want to put our page number in the center.
\cfoot{\footnotesize \thepage} 


%%%%%%%%%%%%%%%%%%%%%%%%%%%%%%%%%%%%%%%%%%%%%%%%
% 4. Your document
%%%%%%%%%%%%%%%%%%%%%%%%%%%%%%%%%%%%%%%%%%%%%%%%

\begin{document}

%----------------------------------------------%
% Title section of the document
%----------------------------------------------%

% For the title section we want to reproduce the title section of the Problem Set and add your names.

\thispagestyle{empty} % This command disables the header on the first page. 

\begin{tabular}{p{15.5cm}} % This is a simple tabular environment to align your text nicely 
    {\large \bf CS2.201: Computer Systems Organization} \\
    Professor: Dr. Praveen Paruchuri                    \\ Spring 2023  \\ International Institute of Information Technology, Hyderabad\\
    \hline % \hline produces horizontal lines.
    \\
\end{tabular} % Our tabular environment ends here.

% \vspace*{0.3cm} % Now we want to add some vertical space in between the line and our title.

\begin{center} % Everything within the center environment is centered.
    {\Large \bf CSO Lab Exam Questions}
    \vspace{2mm}

    % Date or deadline goes here
    {\bf Exam Date: June 10, 2023}

\end{center}
\vspace{0.3cm}

%----------------------------------------------%
% Assignment Introduction %
%----------------------------------------------%

% Welcome to Assignment 1 of the Computer Systems Organisation Course. The aim of this assignment is to familiarize you with writing x86 code. 
% On completion of this assignment you should be able to successfully write arithmetic, conditional, looping components, procedure calls, and conditional jumps in x86-64. 

\paragraph{\textbf{Note:}} Read the given information below carefully.

\begin{itemize}
    \item There are 12 problems in this question bank.
    \item You need to solve only those questions which are assigned to you during the exam.
    \item Assume  signed/unsigned long long int or double based on the question.
          %\item Overflow cases to be handled 
    \item Make suitable assumptions wherever necessary.
    \item Tentative marks problem 1 to 6: 15 marks for each question; from problem 7 to 12: 25 marks for each.
    \item Comments: Not necessary. But some comments to guide us in evaluations would help. Note: Only some basic comments are enough. Don’t add comments for each line. It would take too much of your time and would actually make it difficult for us to find the main parts of your code. So if adding comments, just add a few comments near the main parts of your code.
    \item Naive solution is fine.However, the solution should be reasonable enough. It shouldn't be too complicated too.For example, if you are asked to sort, we don't expect you to use merge sort. You can use a naive algorithm like bubble sort which solves the problem in O($n^2$). But if you come up with something overly complicated like a O($n^{3}$) or O($n^4$) solution and as a result of which your code fails to run even on small/simple test cases, then that can attract penalties.
    \item For some questions we have included function format for convenience but it is not necessary to follow them.
    \item You must strictly stick to the input and output formats.
    \item  Your C file can contain only inputs, outputs and memory allocations. Everything else should be in a function defined in assembly which will be invoked from C file.
    \item No need to handle overflow or invalid input cases unless explicitly asked to handle them in the question.
    \item For the questions requiring sorting, be thorough with the code of Bubble sort, Selection sort and insertion sort. You can be asked to code any of the three without giving choice.


          % \item Make suitable assumptions wherever needed. 

\end{itemize}

% \paragraph{Submission format:}Strictly adhere to the following submission format. Failure to do so may result in an erroneous evaluation of your assignment.

% \begin{itemize}
%     \item The following directory structure is expected, \\
% \hfill\begin{minipage}{\dimexpr\textwidth-3cm}
%  \dirtree{%
% .1 ./$<$\textit{roll\_number$>$}.
% .2 q1.
% .3 q1.s.
% .3 q1.c.
% % .3 stack_helper.c.
% .2 q2.
% .3 q2.s.
% .3 q2.c.
% .2 assignment2.pdf.
% .2 q5.
% % .2 Report.
%  }  
% \xdef\tpd{\the\prevdepth}
% \end{minipage}
%     \item q5 is an executable file.
%     \item Zip the ./$<$\textit{roll\_number$>$} folder and name the zipped folder as $<$\textit{roll\_number}$>$\_assign2.zip

% \end{itemize}

% \clearpage


% \textbf{Assume all the integer variables to be long long int. In case of error or invalid input return '-1'. Overflow cases to be handled (You can use remainder function).}
%----------------------------------------------%
% Problem/Solution section %
%----------------------------------------------%

%add section based on difficulty 


\paragraph{\textcolor{red}{Problem 1: }}
You are a lover of bacteria and you want to raise some bacteria in a box. Initially, the box is empty and each morning, you can put any number of bacteria into the box. And each night, every bacterium in the box will split into two bacterias. You hope to see \textbf{exactly x} bacteria in the box at some moment.
What is the minimum number of bacteria you need to put into the box across those days.?   \\

\vspace*{0.3cm}

\textbf{Input/Output Format}
\begin{itemize}
    \item  Input: X, No of bacterias you want.
    \item  Output: One integer, Answer.
\end{itemize}


\textbf{Sample Test Case}

Input: 5 \\
Output: 2\\

\textbf{Explanation}- For the first sample, we can add one bacterium in the box in the first day morning and at the third morning there will be 4 bacteria in the box. Now we put one more resulting 5 in the box. We added 2 bacteria in the process so the answer is 2.\\

\textbf{Sample Test Case 2}

Input : 8 \\
Output : 1 \\

\vspace*{0.3cm}
%----------------------------------------------%

\paragraph{\textcolor{red}{Problem 2: }}
Given a positive integer N, return an array of integers with all the integers from 1 to N. But for multiples of 3, the array should have -1 instead of the number, for multiples of 5, the array should have -2 instead of the number and for multiples of both 3 and 5, the array should have -3 instead of the number.

\textbf{Input/Output Format}
\begin{itemize}
    \item  Input: N
    \item  Output: N numbers from 1 to N with modifications as required
\end{itemize}


\textbf{Sample Test Case}

Input: 5 \\
Output: 1 2 -1 4 -2\\
\\
Input: 17\\
Output: 1 2 -1 4 -2 -1 7 8 -1 -2 11 -1 13 14 -3 16 17

\vspace*{0.3cm}

%----------------------------%
\paragraph{\textcolor{red}{Problem 3: }}
Given a number N, check if its a palindrome or not.? Palindromes are those numbers which read the same backward and forward. 1, 363, 1331 are palindromes while 10, 456 are not.


\textbf{Input/Output Format}
\begin{itemize}
    \item  Input: N, Single Integer to be checked.
    \item  Output: \textbf{True}, if its a palindrome; \textbf{False}, if not.
    \item  Note: Output case does not matter, TrUe, true, TRUE all are acceptable.
\end{itemize}


\textbf{Sample Test Case}

Input:  13931\\
Output: True\\
\\
Input: 69\\
Output: False\\

\vspace*{0.3cm}

%----------------------------------------------%
\paragraph{\textcolor{red}{Problem 4: }}
Given two integers N and M. Find the GCD of N and M using the Euclidean Algorithm.

\textbf{Input/Output Format}
\begin{itemize}
    \item  Input: N M
    \item  Output: GCD of N and M
    \item  Note: 0 <= M,N <= LONG\_MAX
\end{itemize}

\textbf{Sample Test Case}

Input:  13 3 \\
Output: 1\\

Input : 15 6\\
Output: 3\\
\vspace*{0.3cm}


%----------------------------------------------%
\paragraph{\textcolor{red}{Problem 5: }}
Given an array of binary digits i.e. array consisting only of 0s and 1s, rearrange the elements of array in such a way that all the 0s come before 1s in the array. You need to do this in linear time.\\
\textbf{Hint: } Use Countsort.

\textbf{Input/Output Format}
\begin{itemize}
    \item  Input: Contains two lines. First line has a single integer N, the size of the array; Second line contains N integers where each integer is either 0 or 1
    \item  Output: Rearranged array as required in the question.
\end{itemize}

\textbf{Sample Test Case}

Input:\\
5 \\
0 1 1 0 1  \\
Output:\\
0 0 1 1 1 \\
\\
Input:\\
4\\
0 1 0 0 \\
Output:\\
0 0 0 1 \\

\vspace*{0.3cm}
%----------------------------------------------%
\paragraph{\textcolor{red}{Problem 6: }}
Given a 2-D array of non-negative integers, find the sum of all those integers which are divisible by 2 but not divisible by 3.

\textbf{Input/Output Format}
\begin{itemize}
    \item  Input: M N, where M is number of rows and N is number of columns. Next M lines contain N integers each where $i^{th}$ line represents the elements of  $i^{th}$ row of matrix.
    \item  Output: Single integer which is sum of desired numbers.
\end{itemize}

\textbf{Sample Test Case}

Input:\\
3 3\\
1 2 3\\
4 2 0\\
5 6 8\\
Output: 16\\

\vspace*{0.3cm}

%----------------------------------------------%
\paragraph{\textcolor{red}{Problem 7: }}
Given an array A of size N and a positive integer B, pick \textbf{x} elements from the left end of the array and \textbf{y} elements from the right end of the array, where \textbf{x + y = B}, such that sum of those elements is the \textbf{maximum possible sum} that can be achieved while meeting the above mentioned constraints.\\
\textbf{Note : }0 \leq x,y \leq B.

\textbf{Input/Output Format}
\begin{itemize}
    \item  Input: Has two lines. First line contains two integers N (size of the array) and B. Next line contains N space-separated integers which are elements of the array.
    \item  Output: One integer, maximum possible sum.
\end{itemize}

\textbf{Sample Test Case}

Input:  \\
5 3\\
5 -2 3 1 2 \\
Output: \\
8\\

Input: \\
2 1\\
1 2 \\
Output: \\
2

\vspace*{0.3cm}

%----------------------------------------------%
\paragraph{\textcolor{red}{Problem 8: }}
Given an array of N integers, sort the array into a wave-like array and return it. In other words, arrange the elements into a sequence such that $a_1$ >= $a_2$ <= $a_3$ >= $a_4$ ......\\
\textbf{Note : }If multiple answers are possible, return the lexicographically smallest one.\\
\textbf{Input/Output Format}
\begin{itemize}
    \item  Input: Has two lines. First line contains single integer N, size of the array. Next line contains N space-separated integers which are elements of the array.
    \item  Output: Wave form of input array.
\end{itemize}

\textbf{Sample Test Case}

Input:\\
4\\
1 2 3 4 \\
Output:\\
2 1 4 3\\

Input:\\
2\\
1 2\\
Output:\\
2 1
\vspace*{0.3cm}

%----------------------------------------------%
\paragraph{\textcolor{red}{Problem 9: }}
Given an \textbf{unsorted} array of N integers, find the first missing \textbf{positive} integer.

\textbf{Input/Output Format}
\begin{itemize}
    \item  Input: Has two lines. First line contains a single integer N, size of the array. Next line contains N space-separated integers which are elements of the array.
    \item  Output: Single integer, first missing positive integer
\end{itemize}

\textbf{Sample Test Case}

Input: \\
3\\
2 1 0\\
Output: \\
3\\

Input: \\
4\\
3 4 -1 1\\
Output: \\
2


\vspace*{0.3cm}

%----------------------------------------------%
\paragraph{\textcolor{red}{Problem 10: }}
There are n people standing in line, each looking left or right. Each person counts the number of people in the direction they are looking. The value of the line is sum of each person's count. You are given initial arrangement of people in the line. For \textbf{each k from 1 to n}, determine the maximum value of line if you can change the direction of \textbf{at most} k people.

\textbf{Input/Output Format}
\begin{itemize}
    \item  Input: Has two lines. First line contains a single integer N, size of the array. Next line contains N space-separated integers which are elements of the array. Each integer can either be 0 or 1.\\ \textbf{0 - Looking left, 1 - Looking right}.
    \item  Output: N integers, $i^{th}$ integer is the maximum value of line if you can change direction of atmost i people for each i from 1 to n.
\end{itemize}

\textbf{Sample Test Case}

Input: \\
3\\
0 0 1\\
Output: 3 5 5\\
\\
Input: \\
9\\
0 1 0 1 0 1 0 1 0\\
Output:\\
44 50 54 56 56 56 56 56 56\\


\vspace*{0.3cm}

%----------------------------------------------%
\paragraph{\textcolor{red}{Problem 11: }}
You are given array a of length n. You can perform the following operation as many number of times as you want:\\
Pick two integers i and j ($1 \leq i,j \leq n$) such that $a_i + a_j$ \textbf{is odd}, then swap $a_i$ and $a_j$\\
Output lexicographically  smallest array possible which can be obtained by performing above operation any number of times.\\
\\

\textbf{Input/Output Format}
\begin{itemize}
    \item  Input: Has two lines. First line contains a single integer N, size of the array. Next line contains N space-separated integers which are elements of the array.
    \item  Output: Lexicographically smallest array which can be obtained by performing above operations any number of times.
    \item \textbf{Note: } No need to minimise the number of operations, just  obtain the lexicographically smallest array possible.
\end{itemize}

\textbf{Sample Test Case}

Input: \\ 3 \\ 4 1 7\\
Output:\\ 1 4 7\\

Input: \\ 4 \\ 4 2 6 8 \\
Output:\\ 4 2 6 8 \\

\vspace*{0.3cm}

%----------------------------------------------%
\paragraph{\textcolor{red}{Problem 12: }}
You are asked to take a group photo of 2n people. The $i^{th}$ person has height $h_i$ units. To do so, you need to arrange people in two rows of n people each. To ensure that everyone is seen properly, you must arrange them in such a way that $j_th$ person of the back row must be atleast  x units taller than the $j_th$ person of the front row.

\textbf{Input/Output Format}
\begin{itemize}
    \item  Input: Has two lines. First line contains two integers N (size of the array) and X(The required height difference.). Next line contains 2N space-separated integers which are heights of the people being photographed.
    \item  Output: \textbf{YES}, if such an arrangement exists; \textbf{NO} if no such arrangement exists. Case of Yes and No does not matter. YES, yes, YeS all are acceptable.
\end{itemize}

\textbf{Sample Test Case}

Input:\\
3 6 \\
1 9 3 12 16 10\\
Output:\\
YES\\

Explanation: Front row can contain : 3 1 10 and back row can contain  9  12 16.\\

Input:\\
3 1\\
2 5 2 2 2 5\\
\\
Output:\\
NO\\

Explanation: No such arrangement exists.


\vspace*{0.3cm}

%----------------------------------------------%
\begin{center}
    \textbf{ALL THE BEST}
\end{center}
\end{document}